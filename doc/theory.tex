\documentclass[a4paper,10pt]{article}
%\usepackage{palatino}
\usepackage[english]{babel}
\usepackage{amsfonts}
\usepackage{amsmath}
\usepackage{multirow}
\usepackage[below]{placeins}

% Spelling correction
\newenvironment{centre}{\begin{center}}{\end{center}}

% Formatting
\frenchspacing
\dateenglish
\pagestyle{empty}
\setlength{\parindent}{1em}

\setlength{\topmargin}{-10mm}
\setlength{\oddsidemargin}{0mm}
\setlength{\textwidth}{159mm}
\setlength{\textheight}{246mm}

\newtheorem{theorem}{Theorem}
\newtheorem{corollary}{Corollary}
\newtheorem{lemma}{Lemma}
\newcommand{\QED}{\hfill$\Box$}
\newenvironment{proof}{\textbf{Proof:}}{\QED}

\newcommand{\Bq}{\mathbb{B}_q}
\newcommand{\Bqx}{\mathbb{B}_q^{(x)}}
\newcommand{\Bqxx}{\mathbb{B}_q^{(x+1)}}

\begin{document}

\section{Introduction}

Consider a collection of containers of positive integral size, $c_1
\le c_2 \le \ldots \le c_n)$, with $c_i \ge 0$, and a process which at
each step of the process fills $s$ distinct containers with an item of
size 1. We denote such a problem with the tuple $(\{c_i\}, n, s)$. Let
$t$ be the number of steps taken before there are fewer than $s$
containers with non-zero capacity. We wish to develop an upper bound,
$t_{\max}$ for $t$, and show that the algorithm in section
\ref{algorithm} achieves that bound.

We denote the remaining capacity of containers at the ``current'' step
as $c_i$, and at the following step as $c'_i$. Values at an arbitrary
step $x$ are denoted $c_i^{(x)}$.

\section{Upper bounds on the number of steps}

A trivial upper bound, $t_0$, on $t$ can be found by observing that
each step of the process reduces the total capacity by $s$, and so
\begin{equation}
  \label{t0-bound}
  t_0 = \left\lfloor \frac{\sum c_i}{s} \right\rfloor
\end{equation}
is an upper bound on $t$.

Now, consider a very unbalanced set of containers, where a few of them
($q$, where $q < s$) are much larger than the rest. If we place an
item into each of the the $q$ large containers on every step, then the
remaining $s-q$ items must be distributed within the $n-q$ small
containers. If the large containers are large enough, we will fill all
of the small containers first. We therefore have a reduced problem of
placing $s-q$ items at a time into $n-q$ containers, with a
corresponding trivial bound:
\begin{equation}
  \label{tq-bound}
  t_q = \left\lfloor \frac{\sum_{q<i \le n} c_i}{s-q} \right\rfloor
\end{equation}

This bound holds only if the ``small'' containers in the partition are
sufficiently small. What counts as sufficiently small? We are placing
items in all of the ``large'' containers at each step, so each of
those containers must be larger than $t_q$. Since the $c_i$ are
ordered, this means that for $t_q$ to be a bound on the number of
steps, the smallest ``large'' container is $c_q$, and therefore $t_q$
is a bound if
\begin{equation}
  \label{tq-condition}
  c_q \ge t_q.
\end{equation}

An upper bound on the number of steps possible is therefore the
minimum of the trivial bound and the valid $q$-bounds:
\begin{equation}
  \label{upper-bound-eqn}
  t_{\max} = \min\left(t_0, \min_{q: q < s, c_q \ge t_q} t_q\right).
\end{equation}

Any process which achieves this bound will have placed $s t_{\max}$
items into the containers, and if the bounding constraint is $t_q$
(for $0 \le q < s$), then every container $c_i$ with $1 \le i \le q$
will have $t_{\max}$ items in it. The other containers, with $q+1 \le
i \le n$, will be filled with $(s-q)t_q$ items between them, leaving
$(\sum_{q<i\le n} c_i)-(s-q)t_q$ capacity unfilled.

\section{Optimal algorithm}
\label{algorithm}

Take the $s$ containers with the largest free capacity. Add one item
to each of those. Repeat until there are fewer than $s$ containers
with available capacity. Believed to be optimal, but not proven.

\end{document}
